\section{Introduction}

Many techniques for 2D flow visualization have been developed and
applied. These include grids of little arrows, still the most common
for many applications, equally spaced streamlines
\cite{Turk1996,Jobard1997}, and line integral convolution (LIC)
\cite{Cabral1993}. But which is best and why? \citeN{Laidlaw2001}
showed that the ``which is best'' question can be answered by means
of user studies in which participants are asked to carry out tasks
such as tracing advection pathways or finding critical points in the
flow field. (Note: An advection pathway is the same as a streamline
in a steady flow field.) \citeN{Ware2008} proposed that the ``why''
question may be answered through the application of recent theories
of the way contours in the environment are processed in the visual
cortex of the brain. But Ware only provided a descriptive sketch
with minimal detail and no formal expression. In the present paper,
we show, through a numerical model of neural processing in the
cortex, how the theory predicts which methods will be best for an
advection path tracing task.

% Head 2
\subsection{The IBQ Approach in Image Quality Estimation}

The IBQ approach combined with psychometric methods has proven suitable,
especially for testing the performance of imaging devices or their
components and then returning this quality information to the product
development or evaluation stages. When the subjective changes in image
quality are multivariate, the technical parameters changing in the test
image are unknown or difficult to compute. However, the IBQ approach can be
used to determine the subjectively important quality dimensions with a wide
range of natural image material related to changes caused by different
devices or their components. In order to tune the image-processing
components for optimal performance, it is important to know what the
subjectively crucial characteristics that change in the perceived image
quality are as a function of the tuning parameters, or simply for different
components. Table I describes the problems caused by multivariate changes in
image quality and offers suggestions of how to approach them by using
different measurement methods that complement each other. The IBQ approach
can complement the psychometric approaches and objective measurements by
defining the subjective meaning of image quality attributes and
characteristics; in other words, it reveals how important they are for the
overall perceived quality. This information can then be used as guidance in
tuning, and no complex models are needed in order to understand the relation
between objective measures and subjective quality ratings.

% Table
\begin{table}[t]
\tbl{Multivariate Changes in Image Quality Attributes, the Relationship
of Psychometric and Objective Image Quality Estimations and the IBQ Approach}{%
\begin{tabular}{|l|p{8pc}|p{8pc}|p{12pc}|}
\hline
~PROBLEM   & \multicolumn{3}{l|}{{Estimating the performance when image
                                quality changes are multivariate}}\\\hline
{APPROACH} & {Objective measurements}    & \multicolumn{2}{|{c}|}{Subjective measurements}\\\cline{3-4}
           &                             & IBQ approach          & Psychometric approach\\\hline
GOAL       & Objective and computational
             measures for describing the
             changes in the images       & Definition of
                                           subjectively
                                           crucial image quality
                                           characteristics       & The amount of
                                                                   change in either
                                                                   the overall quality
                                                                   or a single attribute\\\hline
QUESTION   & What changes physically?    & What matters for the
                                           observer?             & How big is the perceived
                                                                   change?\\\hline
\end{tabular}}
\begin{tabnote}
The IBQ approach can help to determine the subjectively crucial
characteristics of an image and therefore to give weights to objective and
computational measures.
\end{tabnote}
\label{tab1}
\end{table}


Our basic rational is as follows. Tracing an advection pathway for a
particle dropped in a flow field is a perceptual task that can be
carried out with the aid of a visual representation of the flow.
The task requires that an individual attempts to trace a continuous
contour from some designated starting point in the flow until some
terminating condition is realized. This terminating condition might
be the edge of the flow field or the crossing of some designated
boundary. If we can produce a neurologically plausible model of
contour perception then this may be the basis of a rigorous theory of
flow visualization efficiency.
% description
\begin{description}
    \item[Identify] Characteristics of an object.
    \item[Locate] Absolute or relative position.
    \item[Distinguish] Recognize as the same or different.
    \item[Categorize] Classify according to some property (e.g.,  color, position, or shape).
    \item[Cluster] Group same or related objects together.
    \item[Distribution] Describe the overall pattern.
    \item[Rank] Order objects of like types.
    \item[Compare] Evaluate different objects with each other.
    \item[Associate] Join in a relationship.
    \item[Correlate] A direct connection.
\end{description}

\subsection{Conditions}
The reproduction of the gestures was performed in the presence or
absence of visual and auditory feedback, resulting in four (2 $\times$ 2) conditions.
% enumerate
\begin{enumerate}
\item Visual and auditory feedback (V\,$+$\,A).
\item Visual feedback, no auditory feedback (V).
\item Auditory feedback, no visual feedback (A).
\item No visual or auditory feedback (None).
\end{enumerate}
The order of the four conditions was randomized across participants.
% itemize
\begin{itemize}
    \item \textit{when} $+$ \textit{where} $\Rightarrow$
          \textit{what}: State the properties of an object or objects at a
          certain ~time, or set of times,  and a certain place, or set of places.
    \item \textit{when} $+$ \textit{what} $\Rightarrow$
          \textit{where}: State the location or set of locations.
    \item \textit{where} $+$ \textit{what} $\Rightarrow$
          \textit{when}: State the time or set of times.
\end{itemize}
When conducting a user study, the goal for the study is to measure
the suitability of the visualization in some sense. What is actually
measured is a fundamental question that we believe can be handled by
using the concepts of {effectiveness}, {efficiency},
and {satisfaction}. These three concepts are derived from the
ISO standard of usability 9241-11.
% quote
\begin{quote}
    Extent to which a product can be used by specified users to
    achieve specified goals with \textit{effectiveness},
    \textit{efficiency}, and \textit{satisfaction} in a specified context of use.
\end{quote}

The mechanisms of contour perception have been studied by
psychologists for at least 80 years, starting with the Gestalt
psychologists. A major breakthrough occurred with the work of Hubel
and Wiesel \citeyear{Hubel1962,Hubel1968} and from that time,
neurological theories of contour perception developed. In this
article, we show that a model of neural processing in the visual
cortex  can be used to predict which flow representation methods will
be better. Our model has two stages. The first is a contour
enhancement model. Contour enhancement is achieved through lateral
connections between nearby local edge detectors. This produces a
neural map in which continuous contours have an enhanced
representation. The model or cortical processing we chose to apply is
adapted from \citeN{Li1998a}. The second stage is a contour
integration model. This represents a higher level cognitive process
whereby a pathway is traced.
% Enunciations
\begin{theorem}
For a video sequence of $n$ frames, an optimal approach based on
dynamic programming can retrieve all levels of key frames together
with their temporal boundaries in O($n^4$) times.
\end{theorem}

We apply the model to a set of 2D flow visualization methods that
were previously studied by \citeN{Laidlaw2001}. This allows us to
carry out a qualitative comparison between the model and how humans
actually performed. We evaluated the model against human performance
in an experiment in which humans and the model performed the same task.

Our article is organized as follows. First we summarize what is
known about the cortical processing of contours and introduce Li's
\citeyear{Li1998a} model of the cortex. Next we show how a slightly
modified version of Li's model differentially enhances various flow
rendering methods. Following this, we develop a perceptual model of
advection tracing and show how it predicts different outcomes for an
advection path-tracing task based on the prior work of
\citeN{Laidlaw2001}. Finally we discuss how this work relates to
other work that has applied perceptual modeling to data visualization
and suggest other uses of the general method.

% Figure
\begin{figure}[tp]
\centering
\includegraphics{./fig/acmlarge-mouse}
\caption{Neurons are arranged in V1 in a column architecture. Neurons
in a particular column respond preferentially to the same edge
orientation. Moving across the cortex (by a minute amount) yields
columns responding to edges having different orientations. A
hypercolumn is a section of cortex that represents a complete set of
orientations for a particular location in space.}
\label{corticalarchitecturefig}
\end{figure}
