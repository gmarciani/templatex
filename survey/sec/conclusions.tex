\section{Conclusions}
The overall agreement between the pattern of results for human
observers and the V1-based model provides strong support of the
perceptual theory we outlined in the introduction. The aligned arrows
style of visualization produced clear chains of mutually reinforcing
neurons along the flow path in the representation, making the flow
pathway easy to trace as predicted by theory.

The fact that LIC produced results as good as the equally spaced
streamlines was something of a surprise, and this lends support to
its popularity within the visualization community. While it did not
produce as much neuron excitation as the aligned arrows method, this
was offset by the lack of nontangential edge responses produced by
glyph-based visualizations. However, its good performance was
achieved only because our evaluation method ignored the directional
ambiguity inherent in this method. \citeN{Laidlaw2001} found this
method to be the worst and there is little doubt that had we allowed
flow in any direction, up or down, human observers would have found
pathways with close to 180 degrees of error half of the time.

The performance of both the model and the human test subjects is
likely to be highly dependent on the underlying vector field used.
As described in Section 5.1.6, the vector field was generated by
interpolating between an 8x8 grid of random, but generally upward
pointing vectors. A consequence of this is that when adjacent vectors
in this grid point somewhat toward each other, the vector field forms
an area of convergence. This convergence area tends to funnel
neighboring streamline paths together, reducing error in streamline
tracing (Figure \ref{regularfig} is an example of this).  Thus, the
overall accuracies of both the model and human subjects may be higher
than might be might be observed using a vector field without such convergence zones.

We were surprised that the computer algorithm actually did better at
the task than human observers. One reason for this may have been that
humans would have to make saccadic eye movements to trace a path,
whereas the computer did not. For the patterns we used, it is likely
that the observers had to make fixations on several successive parts
of a path, and errors may have accumulated as they resumed a trace
from a previous fixation. Nevertheless, we feel that the algorithm
could easily be adjusted to make it give results closer to human
subjects. A more sophisticated approach would be to simulate eye fixations.

The model we applied is a considerable simplification over what
actually occurs. It only uses the simplest model of the simplest
orientation sensitive neurons, and fails to include cortical
magnification, among other shortcomings. Real cortical receptive
fields are not arranged in a rigid hexagonal grid as they are in Li's
model. Furthermore, the neurons of V1 respond to many frequencies,
however our model only uses one in its present form. In addition,
besides the so-called simple cells modeled by \citeN{Li1998a}, other
neurons in V1 and V2 called complex and hypercomplex cells all have
important functions. For example, end-stopped cell respond best to a
contour that terminates in the receptive field and understanding
these may be important in showing how the direction of flow along a
contour can be unambiguously shown. Moreover, visual information is
processed through several stages following the primary cortex,
including V2, V4 and the IT cortex. Each of these appears to abstract
more complex, less localized patterns. Researchers are far from
having sufficient information to model the operations of these stages
all of which may have a role in tracing contours. Nevertheless, the
results are compelling and there are advantages in having a
relatively simple model. We have plans to add some of these more
complex functions in future versions of the model.
