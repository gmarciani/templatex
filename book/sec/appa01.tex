\appendix

\chapter{This Is an Appendix}

Of course, in specifying the particular communication system under
investigation, we must know the important physical parameters,
such as transmitted
power, bandwidth, type(s) of noise present, and so on,
and information theory allows these constraints to be incorporated.
However, information theory does not provide a way for communication system
complexity to be explicitly included.
Although, this is something of a drawback, information theory itself provides
a way around this difficulty, since it is generally true that as we approach
the fundamental limit on the performance of a communication system,
the system complexity increases, sometimes quite drastically.
Therefore, for a simple communication system operating far
from its performance bound, we may be able to improve the performance
with a relatively modest increase in complexity.
On the other hand, if we have a rather complicated communication system
operating near its fundamental limit, any performance improvement may
be possible only with an extremely complicated system.

In this chapter we are concerned with the rather general block diagram
shown in Figure~\ref{ch01.fig11.1.1}. Most of the early work by
Shannon and others ignored the source  encoder/decoder blocks and
concentrated  on bounding the performance of the channel
encoder/decoder pair. Subsequently, the source  encoder/decoder blocks
have attracted much research attention.  In this chapter we consider
both topics and expose the reader to the nomenclature used in the
information theory literature.
Quantitative definitions of information are presented in
Sec.~\ref{ch01.sec11.2} that lay the foundation for the remaining
sections. In Secs.~\ref{ch01.sec11.2} and~\ref{ch01.sec11.1} we present
the fundamental source and channel coding theorems, give some examples,
and state the implications of these theorems.
Section~\ref{ch01.sec11.2} contains a brief development of rate
distortion theory,
which is the mathematical basis for data compression.
A few applications of the theory in this chapter are presented
in Sec.~\ref{ch01.sec11.2}, and a technique for variable-length
source coding is given in Sec.~\ref{ch01.sec11.2}.

In this chapter we are concerned with the rather general block diagram
shown in Figure~\ref{ch01.fig11.1.1}. Most of the early work by
Shannon and others ignored the source  encoder/decoder blocks and
concentrated  on bounding the performance of the channel
encoder/decoder pair. Subsequently, the source  encoder/decoder blocks
have attracted much research attention.  In this chapter we consider
both topics and expose the reader to the nomenclature used in the
information theory literature.
Quantitative definitions of information are presented in
Sec.~\ref{ch01.sec11.2} that lay the foundation for the remaining
sections. In Secs.~\ref{ch01.sec11.2} and~\ref{ch01.sec11.1} we present
the fundamental source and channel coding theorems, give some examples,
and state the implications of these theorems.

\section{What is Classification? What is Prediction?}

Of course, in specifying the particular communication system under
investigation, we must know the important physical parameters,
such as transmitted
power, bandwidth, type(s) of noise present, and so on,
and information theory allows these constraints to be incorporated.
However, information theory does not provide a way for communication system
complexity to be explicitly included.
Although, this is something of a drawback, information theory itself provides
a way around this difficulty, since it is generally true that as we approach
the fundamental limit on the performance of a communication system,
the system complexity increases, sometimes quite drastically.
Therefore, for a simple communication system operating far
from its performance bound, we may be able to improve the performance
with a relatively modest increase in complexity.
On the other hand, if we have a rather complicated communication system
operating near its fundamental limit, any performance improvement may
be possible only with an extremely complicated system.
\begin{enumerate}
\item This is a number list with a short item.
\item And another item that is much longer so that we can make sure it is
formatted correctly and so forth and so on.
\item And a final short item.
\end{enumerate}
